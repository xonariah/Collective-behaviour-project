\documentclass[9pt]{pnas-new}
% Use the lineno option to display guide line numbers if required.
% Note that the use of elements such as single-column equations
% may affect the guide line number alignment. 

%\RequirePackage[english,slovene]{babel} % when writing in slovene
\RequirePackage[slovene,english]{babel} % when writing in english
\DeclareUnicodeCharacter{202F}{ }
\templatetype{pnasresearcharticle} % Choose template 
% {pnasresearcharticle} = Template for a two-column research article
% {pnasmathematics} = Template for a one-column mathematics article
% {pnasinvited} = Template for a PNAS invited submission

\selectlanguage{english}
%\etal{in sod.} % comment out when writing in english
%\renewcommand{\Authands}{ in } % comment out when writing in english
%\renewcommand{\Authand}{ in } % comment out when writing in english

\newcommand{\set}[1]{\ensuremath{\mathbf{#1}}}
\renewcommand{\vec}[1]{\ensuremath{\mathbf{#1}}}
\newcommand{\uvec}[1]{\ensuremath{\hat{\vec{#1}}}}
\newcommand{\const}[1]{{\ensuremath{\kappa_\mathrm{#1}}}} 

\newcommand{\num}[1]{#1}

\graphicspath{{./fig/}}

\title{Swarming behaviour in predator-prey model}

% Use letters for affiliations, numbers to show equal authorship (if applicable) and to indicate the corresponding author
\author{Ariana Kržan}
\author{Tina Brdnik}
\author{Vito Levstik}

\affil{Collective behaviour course research seminar report} 

% Please give the surname of the lead author for the running footer
\leadauthor{Lead author last name} 

\selectlanguage{english}

% Please add here a significance statement to explain the relevance of your work
\significancestatement{Here goes significance statement.}{Simulation | swarming behaviour | predator | prey}

\selectlanguage{english}

% Please include corresponding author, author contribution and author declaration information
%\authorcontributions{Please provide details of author contributions here.}
%\authordeclaration{Please declare any conflict of interest here.}
%\equalauthors{\textsuperscript{1}A.O.(Author One) and A.T. (Author Two) contributed equally to this work (remove if not applicable).}
%\correspondingauthor{\textsuperscript{2}To whom correspondence should be addressed. E-mail: author.two\@email.com}

% Keywords are not mandatory, but authors are strongly encouraged to provide them. If provided, please include two to five keywords, separated by the pipe symbol, e.g:
\keywords{Simulation | swarming behaviour | predator | prey} 

\begin{abstract}
This is a sample abstract. 
\end{abstract}

\dates{\textbf{\today}}
\program{BMA-RI}
\vol{2024/25}
\no{Group G} % group ID
%\fraca{FRIteza/201516.130}


\begin{document}

% Optional adjustment to line up main text (after abstract) of first page with line numbers, when using both lineno and twocolumn options.
% You should only change this length when you've finalised the article contents.
\verticaladjustment{-2pt}

\maketitle
\thispagestyle{firststyle}
\ifthenelse{\boolean{shortarticle}}{\ifthenelse{\boolean{singlecolumn}}{\abscontentformatted}{\abscontent}}{}

% If your first paragraph (i.e. with the \dropcap) contains a list environment (quote, quotation, theorem, definition, enumerate, itemize...), the line after the list may have some extra indentation. If this is the case, add \parshape=0 to the end of the list environment.
\dropcap{T}he sudden emergence of swarming behaviours in animals is one of the most striking examples of collective animal behaviour. 
These behaviours have been extensively studied for their implications for the evolution of cooperation, 
social cognition and predator–prey dynamics\cite{olson2013predator}. Swarming, which appears in many different species like starlings, 
herrings, and locusts, has been linked to several benefits including enhanced foraging efficiency, improved mating success, and distributed problem-solving abilities. 
Furthermore they are hypothesized to help with improving group vigilance, reducing the chance of being encountered by predators, 
diluting an individual's risk of being attacked, enabling an active defence against predators and reducing predator attack efficiency by confusing the predator. \cite{li2023predator}.

In this project we will be taking inspiration from the work of Li et al. (2023) and Olson et al. (2013) to explore how survival pressures can drive the emergence of swarming behaviour. 
The first goal will be to create a realistic simulation where both prey and predators learn to adapt through reinforcement learning based on their drive to survive.
Modeling these interactions, we will observe how simple survival pressures can lead to evolution of more complex behaviours like flocking, swirling and edge predation. 

Then, we will extend our research by evolving out existing model by introducing new environmental obstacles and new species to observe how interspecies interactions lead to new survival strategies.

\section*{Related work}
The modeling of swarming behavior has evolved from foundational rule-based frameworks to more sophisticated reinforcement learning (RL) approaches, with intermediate advances in topological and vision-based models that add realism to agent interactions.

\subsection{Rule-Based models}
Early models of swarming relied on static interaction rules to simulate basic group dynamics. 
Aoki's Zone Model (1982) introduced three interaction zones-repulsion, alignment, and attraction-where agents adjust their behavior based on proximity to neighbors \cite{aoki1987zones}. 
Later, Vicsek's Model (1995) and Reynolds' Boids Model (1987) introduced basic alignment rules (and in Reynolds' case, also cohesion and separation) to generate coordinated group movement \cite{Vicsek1995}\cite{reynolds1987boids}. 
Although effective in modeling simple swarming behaviors, these models rely on fixed rules that limit agents' ability to adapt dynamically to changing environments or threats.

\subsection{Topological and Vision-Based Extensions}

Topological and vision-based models improved realism by adding sensory and neighbor-based constraints. 
Hemelrijk \& Hildenbrandt (2008) introduced a perception model where agents respond only to neighbors that are visible within a variable radius, adjusted by local density, stabilizing cohesion across varied densities \cite{Hemelrijk2008}. 
Kunz \& Hemelrijk (2012) further refined this approach by incorporating visual occlusion, where agents respond only to visible neighbors, simulating real-world sensory limitations \cite{kunz2012}. While these models increase biological realism, they remain rule-based and lack the flexibility of adaptive RL models.

\subsection{Advances in RL-Based Models}

Reinforcement learning has enabled a significant leap in modeling swarming behaviors by allowing agents to learn and adapt based on environmental feedback rather than relying on fixed rules. RL models produce dynamic, emergent behaviors like flocking and evasion, closely mimicking natural adaptive responses to survival pressures.

\begin{itemize}
    \item Li et al. (2023): In this RL-based predator-prey model, prey agents develop swarming behaviors by maximizing survival rewards and minimizing capture penalties \cite{li2023predator}. Prey adaptively form cohesive groups and evade predators, learning these behaviors through experience rather than pre-set alignment rules.

    \item Olson et al. (2013): Olson and colleagues focused on predator confusion, where prey learn to cluster, reducing individual predation risk by confusing predators \cite{olson2013predator}. This model highlights RL's capability to foster adaptive, emergent behaviors in predator-prey dynamics.

    \item Lowe et al. (2017): The Multi-Agent Deep Deterministic Policy Gradient (MADDPG) algorithm enables agents to learn in mixed cooperative and competitive environments, making it ideal for systems with multiple interacting species \cite{lowe2017}. MADDPG supports complex, dynamic environments by allowing agents to optimize strategies in both cooperative and adversarial contexts, such as predators and prey with differing objectives.
\end{itemize}

\subsection{Objective}

Building on the RL framework of Li et al. (2023), this project introduces a third species and environmental obstacles to explore their effects on swarming and predation \cite{li2023predator}. By adding these complexities, we aim to investigate how prey adapt to both predator pressure and environmental challenges, and how interactions with an additional species influence swarm dynamics and survival strategies.


\section*{Methods}

For our project we will be using python and modelling the predator-prey model in a two dimensional space without any edge constrains. 

TODO - 
agent modeling, reinformcnt learning, new spieces and obstacle, evaluation metric?


\section*{Results}
Even though we currently don't have any results, we can outline what we anticipate to achieve. One of the results will be a simple model that is a recreation of Li et al. (2023) model in python. It is expected to model the development of swarming behaviours.  We have already extracted their code and have successful ran it. We intend to compare our results with the article.

After that, we will build upon our existing model by adding new obstacles and interspecies interactions. 

\section*{Discussion}
Though we just started working on the project, we have already encountered some issues when trying to run the code of our main article. We managed to contact the original authors and they were quick and happy to help so as of right now, we are happy to report we got the code working and running. 

We acknowledge that we have set ambitious goals, especially by introducing a new species into the mix. We anticipate a lot of new issues will arise through the semester, however we are excited to challenge ourselves and we are determined to learn from the process no matter the outcome.

Initially this project began with just two people working on it, but due to the complex nature of the project, we have joined forces with another group of just one member. This will enable us to work on the assignment more efficiently and have easier time achieving our goals.


\acknow{AK worked on introduction, expected results and discussion, TB will do that, VL will do this and that.}
\showacknow % Display the acknowledgments section

% \pnasbreak splits and balances the columns before the references.
% If you see unexpected formatting errors, try commenting out this line
% as it can run into problems with floats and footnotes on the final page.
%\pnasbreak

\begin{multicols}{2}
	%\bibliographystyle{plain} % or any style you prefer, such as ieee, unsrt, etc.
	\bibliography{bib/bibliography} % This assumes your .bib file is located at ./bib/bibliography.bib
\end{multicols}


\end{document}